%%------------------------------------------------------
% 
%% UNIVERSIDADE FEDERAL DE SANTA CATARINA - UFSC
%
%% Prof.: Wyllian B. da Silva
%%
%% Template: estilo IEEEtran [paper com duas colunas]
%% Adaptado de: https://ieeeauthorcenter.ieee.org/create-your-ieee-article/use-authoring-tools-and-ieee-article-templates/ieee-article-templates/templates-for-transactions/
%               https://ctan.org/tex-archive/macros/latex/contrib/IEEEtran?lang=en

%% Instruções: http://mirrors.ctan.org/macros/latex/contrib/IEEEtran/IEEEtran_HOWTO.pdf
%%
%% Recomendações:
%% Utilize o Edito Kile (SO Linux)
%% Certifique-se de que a codificação de caracteres utilizada é a UTF-8





\documentclass[journal]{IEEEtran}


%%------------------------------------------------------
%% Packages
%%------------------------------------------------------

\usepackage[utf8]{inputenc}        %% Codificação de caracteres (conversão automática dos acentos)
\usepackage[dvips]{graphicx}       %% para a macro includegraphics 
\usepackage[english,brazil]{babel} %% PT_BR e EN (o último define a prioridade no arquivo)
\usepackage{pgf}                   %% macro para criar gráficos
\usepackage{epsfig}                %% or use the epsfig package if you prefer to use the old commands
\usepackage{graphics}              %% use the graphics package for simple commands
\usepackage{graphicx}              %% or use the graphicx package for more complicated commands
\usepackage{epstopdf} 
\usepackage{listings}



\usepackage{hyperref}              %% enalbe one-click link

% \usepackage{showframe} %% just for the example

% \usepackage[sort,compress]{cite} %% disable if natbib package is activated




%%------------------------------------------------------
%% Definitions
%%------------------------------------------------------

\hyphenation{op-tical net-works semi-conduc-tor}

%% Can use something like this to put references on a page
%% by themselves when using endfloat and the captionsoff option.
\ifCLASSOPTIONcaptionsoff
  \newpage
\fi


%%----------------- Definindo as variáveis com números
\makeatletter
%
\newcommand{\prenome}{\afterassignment\prenome@aux\count0=}
\newcommand{\prenome@aux}{\csname prenome\the\count0\endcsname}
%
\newcommand{\nomedomeio}{\afterassignment\nomedomeio@aux\count0=}
\newcommand{\nomedomeio@aux}{\csname nomedomeio\the\count0\endcsname}
%
\newcommand{\sobrenome}{\afterassignment\sobrenome@aux\count0=}
\newcommand{\sobrenome@aux}{\csname sobrenome\the\count0\endcsname}
\makeatother
%%%%%

%%----------------- Configurações de hyperlinks
%% Não decorado, sem destaque
\hypersetup{
  colorlinks=false,
  pdfborder={0 0 0},
}




%%------------------------------------------------------
%% Configurações
%%------------------------------------------------------

%%----------------- Título
\title                                                {PROJETO GAME-THE SCAPIST}{}
\author{
  Jaqueline Cristina da Rosa  \thanks{Use footnote for providing further
    information about author (webpage, alternative
    address)---\emph{not} for acknowledging funding agencies.} \\
  Bacharelado em Ciências e Tecnologia \\
  Universidade Federal de Santa Catarina\\
  jaqueline_rosa06@hotmail.com\\
  \texttt{}
  %% examples of more authors
   \And
 \\Pedro Lucas Sousa Gonçalves \\
  Engenharia Ferroviária e Metroviária\\
  Universidade Federal de Santa Catarina\\
  pedrolucas.2713.sg@gmail.com \\
  \texttt{}
  %% \AND
  %% Coauthor \\
  %% Affiliation \\
  %% Address \\
  %% \texttt{email} \\
  %% \And
  %% Coauthor \\
  %% Affiliation \\
  %% Address \\
  %% \texttt{email} \\
  %% \And
  %% Coauthor \\
  %% Affiliation \\
  %% Address \\
  %% \texttt{email} \\
}

\newcommand{\emailautor}                              {seuemail@grad.ufsc.br}

\newcommand{\siglaRevista}                            {}

\newcommand{\Revista}                                 {Universidade Federal de Santa Catarina (UFSC)}







% %%----------------- Vários Autores
% \author{\IEEEauthorblockN{\prenomePrincipal~\nomedomeioPrincipal~\sobrenomePrincipal\IEEEauthorrefmark{1}, 
% \prenome2~\nomedomeio2~\sobrenome2\IEEEauthorrefmark{2}, 
% \prenome3~\nomedomeio3~\sobrenome3\IEEEauthorrefmark{3}, 
% \prenome4~\nomedomeio4~\sobrenome4\IEEEauthorrefmark{3}, and 
% \prenome5~\nomedomeio5~\sobrenome5\IEEEauthorrefmark{4},~\IEEEmembership{Fellow,~IEEE}}
% 
% \IEEEauthorblockA{\IEEEauthorrefmark{1}Universidade Federal de Santa Catarina (UFSC)}
% 
% \IEEEauthorblockA{\IEEEauthorrefmark{2}School of Electrical and Computer Engineering, Georgia Institute of Technology, Atlanta, GA 30332 USA}
% 
% \IEEEauthorblockA{\IEEEauthorrefmark{3}Starfleet Academy, San Francisco, CA 96678 USA}
% 
% \IEEEauthorblockA{\IEEEauthorrefmark{4}Tyrell Inc., 123 Replicant Street, Los Angeles, CA 90210 USA}% <-this % stops an unwanted space
% %%
% \thanks{\Revista~(\siglaRevista). Correspond\^encia ao autor: \prenomePrincipal~\nomedomeioPrincipal~\sobrenomePrincipal~(email: \emailautor).}}



%%------------------------------------------------------
%% Cabeçalho
\markboth{\MakeUppercase{\Revista}}%% acentuação indireta
%% Apenas um autor:
{\sobrenomePrincipal: \MakeUppercase{\Revista}}
%% Mais de um autor:
% {\sobrenomePrincipal: \MakeLowercase{\textit{et al.}}: \Revista}




 






\begin{document}



%%------------------------------------------------------
%% Inserção de informações
\maketitle
\IEEEdisplaynontitleabstractindextext
\IEEEpeerreviewmaketitle


%%------------------------------------------------------
%% Section
\section{Introdução}
Este documento tem por objetivo descrever o desenvolvimento da criação de um jogo, baseado nas aulas lecionadas pelo professor Wyllian Bezerra da Silva da matéria de Programação I. O jogo foi criado a partir dos jogos PacMan e The Escapist, sendo aproveitado detalhes diferentes de cada um dos jogos. O jogo foi desenvolvido utilizando a linguagem de programação C. Para a execução deste trabalho apenas é necessário um computador com máquina virtual Ubuntu instalada, e em correto funcionamento. 

%%------------------------------------------------------
%% Section
\section{Pac man}
A mecânica do jogo é simples: o jogador é uma cabeça redonda com uma boca que se abre e fecha, posicionado em um labirinto simples repleto de pastilhas e 4 fantasmas que o perseguem. O objetivo é comer todas as pastilhas sem ser alcançado pelos fantasmas, em ritmo progressivo de dificuldade. 


%%--------------------------------------------------------
%% Section
\section{The Escapists}
The Escapists é uma simulação de fuga de presídio premiada que dá ao jogador a oportunidade de experimentar de forma divertida a vida diária na prisão. O objetivo do jogo é simplesmente fugir. 

%%---------------------------------------------------------
%% Section
\section{O Jogo}
O jogo consiste em um labirinto de três fases. O objetivo do jogo é passar as três fases no menor tempo possível, e sem tocar no inimigo que se move de forma pseudo aleatória. O jogo não possui interface gráfica, apenas caracteres/símbolos no terminal, a cada movimento do personagem é emitido um beep. No fim da terceira fase o tempo é imprimido na tela indicando os segundos que passaram para percorrer as três fases. 
 
%%-------------------------------------------------------------
%% Section
\section{Movimentação do peronagem}
O botão A, responsável por mexer nosso personagem para a esquerda, mudara no eixo X do personagem decrementando -1 de j. O botão D, responsável por mexer nosso personagem para a direita, mudara no eixo X do personagem incrementando 1 no j. O botão W, responsável por mexer nosso personagem para cima, mudara no eixo Y do personagem decrementando -1 em i. O botão S, responsável por mexer nosso personagem para baixo, mudara no eixo Y do personagem  incrementando 1 no i. Deve-se levar em consideração que quando mandamos o personagem para uma nova posição, quando for uma parede ele continua na mesma posição e quando for a saída ele muda de fase ou ganha o jogo. 

\begin{lstlisting}  
        if (movimento == 'w') {  
          if (mapa[i-1][j]==0){ 
           system("\abeep"); 
           mapa[i][j]=0; 
           i = i-1; 
         }else if (mapa[i-1][j]==1){    } 
          else if (mapa[i-1][j]==2){ 
          return 1;
          }
          }
          \end{lstlisting}
%% ----------------------------------------------------------
%% Section
\section{Movimentação do inimigo}
Utilizando a mesma lógica que foi utilizada para movimentar o personagem, foi possível criar a movimentação do inimigo. Para mudar a posição do personagem utilizamos as teclas: w, a, s, d, que indicam respectivamente: para cima, esquerda, direita, para baixo. Na movimentação do inimigo usamos a função rand(), que é responsável por gerar números aleatórios. Através dela geramos 4 números de 0 a 3, cada um desses 4 números gerar um movimento diferente. Através do número 0 o inimigo de move para cima, 1 o inimigo se move para baixo, 2 o inimigo se move para a direita e 3 o inimigo de move para a esquerda. 

\begin{lstlisting}  
 int moveEnemy(){
 char movimentoE= rand()%4;
   if(movimentoE==0){
    if(mapa[x-1][y]==0){
      mapa[x][y]=0;
      x=x-1;
  }else if ((mapa[x-1][y]==1)||(mapa[x-1][y]==2)){
     }
   }
}
         \end{lstlisting}
         
%% ----------------------------------------------------------
%% Section
\section{Resultados}
Através desse projeto é nítido a evolução no conhecimento em relação a matéria de Programação I, pois além do conteúdo visto em aula, foi necessário a pesquisa de alguns itens até então não conhecidos. O jogo atendeu cerca de 95\% do que foi proposto na descrição inicial que foi postada, o que é satisfatório. 

%%--------------------------------------------------
%% Section

\section{Conclusão}
A utilização de recursos da programação nos permite, criar de programas simples a jogos complexos, além de nos ajudar a realizar nossos trabalhos de forma mais rápida e eficiente. A partir desse projeto foi possível aprender na prática toda teoria passada durante as aulas de programação I, e a buscar conhecimento além do que nos foi imposto. 





%%------------------------------------------------------
%% Section (no numbering, use section* for acknowledgment)
%% UFSC: no necessary
% \section*{Acknowledgment}
% The authors would like to thank...



%%------------------------------------------------------
%% References (Option 1): extern file
%% Edit with JabRef, for instance
\bibliographystyle{IEEEtran} %% Estilo da referência
\bibliography{references}    %% Caminho do arquivo (sem extensão)



% %%------------------------------------------------------
% %% References (Option 2): incorporeted
% 
% \begin{thebibliography}{10} 
% 
%   \bibitem{Kopka:1999}
%     H.~Kopka and P.~W. Daly, \emph{A Guide to
%     \LaTeX}, 3rd~ed. Harlow, 
%     England: Addison-Wesley, 1999.
% 
%   \bibitem{Huynen:1998} 
%     Huynen, M.~A. and Bork, P. 1998. Measuring 
%     genome evolution. {\em Proceedings of the 
%     National Academy of Sciences USA} 
%     95:5849--5856.
% 
%   \bibitem{Caprara:1997} 
%     Caprara, A. 1997. Sorting by reversals is 
%     difficult. In: {\em Proceedings of the 
%     First Annual International Conference on 
%     Computational Molecular Biology 
%     (RECOMB 97),} New York: ACM.  pp. 75-83.
% 
%   \bibitem{Reinelt:1991}
%     Reinelt, G. 1991. {\em The Traveling 
%     Salesman -- Computational Solutions for TSP 
%     Applications.} Berlin: Springer Verlag.
% 
% \end{thebibliography}



\end{document}
