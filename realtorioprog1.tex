%%------------------------------------------------------
% 
%% UNIVERSIDADE FEDERAL DE SANTA CATARINA - UFSC
%
%% Prof.: Wyllian B. da Silva
%%
%% Template: estilo IEEEtran [paper com duas colunas]
%% Adaptado de: https://ieeeauthorcenter.ieee.org/create-your-ieee-article/use-authoring-tools-and-ieee-article-templates/ieee-article-templates/templates-for-transactions/
%               https://ctan.org/tex-archive/macros/latex/contrib/IEEEtran?lang=en

%% Instruções: http://mirrors.ctan.org/macros/latex/contrib/IEEEtran/IEEEtran_HOWTO.pdf
%%
%% Recomendações:
%% Utilize o Edito Kile (SO Linux)
%% Certifique-se de que a codificação de caracteres utilizada é a UTF-8





\documentclass[journal]{IEEEtran}


%%------------------------------------------------------
%% Packages
%%------------------------------------------------------

\usepackage[utf8]{inputenc}        %% Codificação de caracteres (conversão automática dos acentos)
\usepackage[dvips]{graphicx}       %% para a macro includegraphics 
\usepackage[english,brazil]{babel} %% PT_BR e EN (o último define a prioridade no arquivo)
\usepackage{pgf}                   %% macro para criar gráficos
\usepackage{epsfig}                %% or use the epsfig package if you prefer to use the old commands
\usepackage{graphics}              %% use the graphics package for simple commands
\usepackage{graphicx}              %% or use the graphicx package for more complicated commands
\usepackage{epstopdf}              %% enable EPS (convert to PDF)



\usepackage{hyperref}              %% enalbe one-click link

% \usepackage{showframe} %% just for the example

% \usepackage[sort,compress]{cite} %% disable if natbib package is activated




%%------------------------------------------------------
%% Definitions
%%------------------------------------------------------

\hyphenation{op-tical net-works semi-conduc-tor}

%% Can use something like this to put references on a page
%% by themselves when using endfloat and the captionsoff option.
\ifCLASSOPTIONcaptionsoff
  \newpage
\fi


%%----------------- Definindo as variáveis com números
\makeatletter
%
\newcommand{\prenome}{\afterassignment\prenome@aux\count0=}
\newcommand{\prenome@aux}{\csname prenome\the\count0\endcsname}
%
\newcommand{\nomedomeio}{\afterassignment\nomedomeio@aux\count0=}
\newcommand{\nomedomeio@aux}{\csname nomedomeio\the\count0\endcsname}
%
\newcommand{\sobrenome}{\afterassignment\sobrenome@aux\count0=}
\newcommand{\sobrenome@aux}{\csname sobrenome\the\count0\endcsname}
\makeatother
%%%%%

%%----------------- Configurações de hyperlinks
%% Não decorado, sem destaque
\hypersetup{
  colorlinks=false,
  pdfborder={0 0 0},
}




%%------------------------------------------------------
%% Configurações
%%------------------------------------------------------

%%----------------- Título
\title                                                {PROJETO GAME-THE SCAPIST}{}
\author{
  Jaqueline Cristina da Rosa  \thanks{Use footnote for providing further
    information about author (webpage, alternative
    address)---\emph{not} for acknowledging funding agencies.} \\
  Bacharelado em Ciências e Tecnologia \\
  Universidade Federal de Santa Catarina\\
  jaqueline_rosa06@hotmail.com\\
  \texttt{}
  %% examples of more authors
   \And
 \\Pedro Lucas Sousa Gonçalves \\
  Engenharia Ferroviária e Metroviária\\
  Universidade Federal de Santa Catarina\\
  pedrolucas.2713.sg@gmail.com \\
  \texttt{}
  %% \AND
  %% Coauthor \\
  %% Affiliation \\
  %% Address \\
  %% \texttt{email} \\
  %% \And
  %% Coauthor \\
  %% Affiliation \\
  %% Address \\
  %% \texttt{email} \\
  %% \And
  %% Coauthor \\
  %% Affiliation \\
  %% Address \\
  %% \texttt{email} \\
}

\newcommand{\emailautor}                              {seuemail@grad.ufsc.br}

\newcommand{\siglaRevista}                            {}

\newcommand{\Revista}                                 {Universidade Federal de Santa Catarina (UFSC)}







% %%----------------- Vários Autores
% \author{\IEEEauthorblockN{\prenomePrincipal~\nomedomeioPrincipal~\sobrenomePrincipal\IEEEauthorrefmark{1}, 
% \prenome2~\nomedomeio2~\sobrenome2\IEEEauthorrefmark{2}, 
% \prenome3~\nomedomeio3~\sobrenome3\IEEEauthorrefmark{3}, 
% \prenome4~\nomedomeio4~\sobrenome4\IEEEauthorrefmark{3}, and 
% \prenome5~\nomedomeio5~\sobrenome5\IEEEauthorrefmark{4},~\IEEEmembership{Fellow,~IEEE}}
% 
% \IEEEauthorblockA{\IEEEauthorrefmark{1}Universidade Federal de Santa Catarina (UFSC)}
% 
% \IEEEauthorblockA{\IEEEauthorrefmark{2}School of Electrical and Computer Engineering, Georgia Institute of Technology, Atlanta, GA 30332 USA}
% 
% \IEEEauthorblockA{\IEEEauthorrefmark{3}Starfleet Academy, San Francisco, CA 96678 USA}
% 
% \IEEEauthorblockA{\IEEEauthorrefmark{4}Tyrell Inc., 123 Replicant Street, Los Angeles, CA 90210 USA}% <-this % stops an unwanted space
% %%
% \thanks{\Revista~(\siglaRevista). Correspond\^encia ao autor: \prenomePrincipal~\nomedomeioPrincipal~\sobrenomePrincipal~(email: \emailautor).}}



%%------------------------------------------------------
%% Cabeçalho
\markboth{\MakeUppercase{\Revista}}%% acentuação indireta
%% Apenas um autor:
{\sobrenomePrincipal: \MakeUppercase{\Revista}}
%% Mais de um autor:
% {\sobrenomePrincipal: \MakeLowercase{\textit{et al.}}: \Revista}




 






\begin{document}



%%------------------------------------------------------
%% Inserção de informações
\maketitle
\IEEEdisplaynontitleabstractindextext
\IEEEpeerreviewmaketitle


%%------------------------------------------------------
%% Section
\section{Introdução}
A linguagem utilizada para a programação do jogo foi C,  sendo utilizado o linux para a implementação do código.O  jogo consiste em um labirinto de três fases. O objetivo do jogo é passar as três fases no menor tempo possível, e sem tocar no inimigo que se move de forma pseudo aleatória. O jogo não possui interface gráfica,apenas caracteres/símbolos no terminal,a cada movimento do personagem  é emitido um beep. No fim da terceira fase o tempo é imprimido na tela indicando os segundos  que passaram para percorrer as três fases.









%%------------------------------------------------------
%% Section
\section{Metodologia}
Para jogar é utilizado apenas o teclado, sendo a movimentação da seguinte forma: 'w' :movimento para cima, 's': movimento para baixo, 'd': movimento para direita e 'a': movimento para esquerda. A tecla 'x' é utilizada para sair do jogo, já a tecla 'r' para reiniciar.


\section{Codificação estruturada}
\begin{verbatim}
#include <stdio.h>
#include <stdlib.h>
#include <time.h>
#include <string.h>
#include <unistd.h>

int i= 1;
int j= 0;
int x=18;
int y=12;
int mapa[20][16];
char fase=1;
time_t seconds;
time_t actual;

int mapa1[20][16] = {
{1,1,1,1,1,1,1,1,1,1,1,1,1,1,1,1},
{1,0,0,0,0,0,0,0,0,0,0,0,0,1,1,1},
{1,0,0,0,0,0,0,1,1,0,0,0,0,1,1,1},
{1,0,0,0,0,0,0,1,1,0,0,0,0,1,1,1},
{1,0,0,0,0,0,0,1,1,0,0,0,0,1,1,1},
{1,0,0,0,0,0,0,1,1,0,0,0,0,1,1,1},
{1,0,0,0,0,0,0,0,0,0,0,0,0,0,1,1},
{1,1,1,1,1,1,1,1,1,1,1,1,1,0,1,1},
{1,1,1,1,1,1,1,1,1,1,1,1,1,0,1,1},
{1,1,0,0,0,0,0,0,0,0,0,0,0,0,1,1},
{1,1,0,1,1,1,1,0,1,1,1,1,1,1,1,1},
{1,0,0,0,0,0,0,0,0,0,0,0,0,0,1,1},
{1,0,1,1,1,1,1,0,1,1,1,1,1,0,1,1},
{1,0,1,1,1,1,1,0,0,1,1,1,1,0,1,1},
{1,0,0,1,1,1,1,1,0,1,1,1,1,0,1,1},
{1,0,0,1,1,1,1,1,0,0,0,0,0,0,1,1},
{1,1,0,1,1,1,1,1,1,1,1,1,0,1,1,1},
{1,0,0,0,0,0,0,0,1,1,1,1,0,1,1,1},
{1,1,1,1,1,1,1,0,0,0,0,0,0,0,2,1},
{1,1,1,1,1,1,1,1,1,1,1,1,1,1,1,1}

 };
int mapa2[20][18] = {
{1,1,1,1,1,1,1,1,1,1,1,1,1,1,1,1},
{1,0,0,0,0,0,0,0,0,0,0,0,1,1,1,1},
{1,0,0,0,1,1,0,0,0,0,0,0,1,1,1,1},
{1,0,0,0,1,1,0,0,0,0,0,0,1,1,1,1},
{1,0,0,0,0,0,0,0,0,0,0,0,0,0,0,1},
{1,1,1,1,1,1,0,1,1,1,1,1,1,1,0,1},
{1,1,1,1,1,1,0,1,1,1,1,1,1,1,0,1},
{1,0,0,0,0,0,0,0,0,0,0,0,0,0,0,1},
{1,0,1,1,1,1,1,1,1,1,1,0,1,1,0,1},
{1,0,1,1,1,1,1,1,1,1,1,0,1,1,0,1},
{1,0,0,0,0,0,0,0,0,0,0,0,0,0,0,1},
{1,1,0,1,1,1,1,1,1,0,1,1,1,0,1,1},
{1,1,0,1,1,1,1,1,1,0,1,1,1,0,1,1},
{1,0,0,0,0,0,0,0,0,0,1,1,1,0,1,1},
{1,0,1,1,1,1,0,1,1,1,1,1,1,0,1,1},
{1,0,1,1,1,1,0,1,1,1,1,1,1,0,1,1},
{1,0,1,1,1,1,0,0,0,0,0,0,0,0,1,1},
{1,0,1,1,1,1,0,1,1,0,1,1,1,1,1,1},
{1,0,0,0,0,0,0,1,1,0,0,0,0,0,2,1},
{1,1,1,1,1,1,1,1,1,1,1,1,1,1,1,1}
 };
 int mapa3[20][16] = {
{1,1,1,1,1,1,1,1,1,1,1,1,1,1,1,1},
{1,0,0,0,0,0,0,0,0,0,0,0,0,0,0,1},
{1,0,0,0,0,0,0,0,0,0,0,0,0,0,0,1},
{1,0,0,0,0,0,0,0,0,0,0,0,0,0,0,1},
{1,1,1,1,1,1,0,1,1,1,1,1,1,1,0,1},
{1,1,1,1,1,1,0,1,1,1,1,1,1,1,0,1},
{1,0,0,0,0,0,0,0,0,0,0,0,0,0,0,1},
{1,0,1,1,1,1,1,1,0,1,1,1,1,1,0,1},
{1,0,0,0,0,0,0,0,0,1,1,1,1,1,0,1},
{1,1,1,0,1,1,1,1,1,1,1,1,1,1,0,1},
{1,1,1,0,1,1,1,1,1,1,1,1,1,1,0,1},
{1,1,1,0,1,1,1,1,1,1,1,1,1,1,0,1},
{1,0,0,0,0,0,0,0,0,0,0,0,0,0,0,1},
{1,0,1,1,1,1,1,1,1,1,1,1,1,1,0,1},
{1,0,0,0,0,0,0,0,0,1,1,1,1,1,0,1},
{1,1,1,0,1,1,1,1,0,0,0,0,0,0,0,1},
{1,1,1,0,0,0,1,1,0,1,1,1,1,1,1,1},
{1,1,1,1,1,0,1,1,0,1,1,1,1,1,1,1},
{1,1,1,1,1,0,0,0,0,0,0,0,0,0,2,1},
{1,1,1,1,1,1,1,1,1,1,1,1,1,1,1,1}
 };

void labirinto()
 {
   int linha,coluna;

if(fase==1){
for (linha=0;linha<20;linha++) {
for (coluna=0;coluna<16;coluna++) {
 mapa[linha][coluna]=mapa1[linha][coluna];
    }
    }
}
if(fase==2){
for (linha=0;linha<20;linha++) {
for (coluna=0;coluna<16;coluna++) {
mapa[linha][coluna]=mapa2[linha][coluna];
           }
         }
       }
if (fase==3){
for (linha=0;linha<20;linha++) {
for (coluna=0;coluna<16;coluna++) {
mapa[linha][coluna]=mapa3[linha][coluna];
            }
          }
       }
for (linha=0;linha<20;linha++) {
for (coluna=0;coluna<16;coluna++) {
if ((linha == i) && (coluna == j)) {
printf("\033[1;35m");
printf("^");
printf("\033[0m");
continue;
    }
if ((linha == x) && (coluna == y)) { 
printf("\033[1;35m");
printf("▓");
printf("\033[0m");
continue;
 }
if (mapa[linha][coluna] == 0) 
printf(" ");
if (mapa[linha][coluna] == 1)
{printf("\033[1;31m");
printf("▓");
printf("\033[0m");}
              
if (mapa[linha][coluna] == 2)
{printf("\033[1;32m");
printf("▓");
printf("\033[0m");}
 }
printf("\n");
         }
    }
int move_enemy(){
char movimentoE= rand()%4;
if(movimentoE==0){ 
if(mapa[x-1][y]==0){
mapa[x][y]=0;
x=x-1;
}else if ((mapa[x-1][y]==1)||(mapa[x-1][y]==2)){

}
}
if(movimentoE==1){
if(mapa[x+1][y]==0){ 
 mapa[x][y]=0;
x=x+1;
}else if ((mapa[x+1][y]==1)||(mapa[x+1][y]==2)){
            }
    }
 if(movimentoE==2){
 if(mapa[x][y-1]==0){
    mapa[x][y]=0;
    y=y-1;
 }else if ((mapa[x][y-1]==1)||(mapa[x][y-1]==2)){
  }
 }
 if(movimentoE==3){
    if(mapa[x][y+1]==0){
     mapa[x][y]=0;
        y=y+1;
        }else if (mapa[x][y+1]==1){}
         else if (mapa[x][y+1]==2){}
    }
     if(i == x && j == y){
     system("clear");
     char debug; i=1;j=0;x=1;y=2; 
     fase = 1; 
     while(debug != 'r'){
     scanf("%c",&debug); 
     system("clear");
     printf("FIM DE JOGO");                                         if(debug == 'x')
       exit(0);} }
}


int move(char movimento)
{if (movimento == 'w') { 
if (mapa[i-1][j]==0){
system("\abeep");
mapa[i][j]=0;
i = i-1;

 }else if (mapa[i-1][j]==1){   
     }
else if (mapa[i-1][j]==2){
return 1;
     }
}
if (movimento == 's')  {
if (mapa[i+1][j]==0){
system("\abeep");
mapa[i][j]=0;
i= i+1;
}else if(mapa[i+1][j]==1){
}else if(mapa[i+1][j]==2){
 return 1;
 }
 }
if (movimento == 'd') {
 if(mapa[i][j+1]==0){
system("\abeep");
mapa[i][j]=0;
j=j+1;
}else if(mapa[i][j+1]==1){
}else if (mapa[i][j+1]==2){
return 1;
 }
 }
if (movimento == 'a') {
if(mapa[i][j-1]==0){
system("\abeep");
 mapa[i][j]=0;
 j=j-1;
 }else if (mapa[i][j-1]==1){

}else if(mapa[i][j-1]==2){
system("clear");
          
 return 1;
        }
 }
    if(i == x && j == y){
    system("clear");
    char debug; 
    i=1;j=0;x=18;y=12; 
    fase = 1; 
    while(debug != 'r'){
    scanf("%c",&debug); 
    system("clear");
    printf("FIM DE JOGO");                                          if(debug == 'x') 
    exit(0);}seconds = time(NULL); }
	return 0;

}






void imprime_inicio()

{
system("clear");
printf("                                                                                                               \n"); usleep (100000);
printf("   ###############  ###         ###  ###############                                                           \n"); usleep (100000); printf("\033[0;32m");
printf("   ###############  ###         ###  ###############                                                           \n"); usleep (100000); printf("\033[0;33m");
printf("         ###        ###         ###  ###                                                                       \n"); usleep (100000); printf("\033[0;34m");
printf("         ###        ###############  ##########                                                                \n"); usleep (100000); printf("\033[0;35m");
printf("         ###        ###############  ##########                                                                \n"); usleep (100000);
printf("\033[1;35m");
printf("         ###        ###         ###  ###                                                                       \n"); usleep (100000);
printf("\033[0;36m");
printf("         ###        ###         ###  ###                                                                       \n"); usleep (100000);
printf("\033[1;32m");
printf("         ###        ###         ###  ###############                                                           \n"); usleep (100000);
printf("\033[0;31m");
printf("         ###        ###         ###  ###############                                                           \n"); usleep (100000);
printf("                                                                                                               \n"); usleep (100000);
printf("   ###############  ###############  ###############  ###############  ###  ###############  ###############   \n"); usleep (100000);
printf("\033[0;32m");
printf("   ###############  ###############  ###############  ###############  ###  ###############  ###############   \n"); usleep (100000);
printf("\033[0;33m");
printf("   ###              ###              ###         ###  ###         ###  ###  ###                    ###         \n"); usleep (100000);
printf("\033[0;34m");
printf("   ###############  ###              ###############  ###############  ###  ###############        ###         \n"); usleep (100000);
printf("\033[0;35m");
printf("   ###############  ###              ###############  ###############  ###  ###############        ###         \n"); usleep (100000);
printf("               ###  ###              ###         ###  ###              ###              ###        ###         \n"); usleep (100000);
printf("\033[0;31m");
printf("               ###  ###              ###         ###  ###              ###              ###        ###         \n"); usleep (100000);
printf("\033[1;35m");
printf("   ###############  ###############  ###         ###  ###              ###  ###############        ###         \n"); usleep (100000);
printf("\033[0;36m");
printf("   ###############  ###############  ###         ###  ###              ###  ###############        ###         \n"); usleep (100000);
printf("\033[0;31m");
printf("                                                                                                               \n"); usleep (100000);

}

int timer()
{
	actual = time(NULL);
	int now = (int)difftime(actual,seconds);
	return now;
}
    int main(){
    char movimento;	
    seconds = time(NULL);
    imprime_inicio();
    while (1){
    system("clear");
    printf("\t      Time:%d\n", timer());
    move_enemy();//
    labirinto();
    printf("\n Aperte (x) para sair do Jogo");
    scanf("%c", &movimento);
    if(movimento=='x') break;
         else if(move(movimento) == 1){
          fase+=1;
          i=1;
          j=0;	
   if(fase>3){
   printf("tempo:%d segundos", timer());
      break;
	}
      }
    }

    return(0);
}
\end{verbatim}



%%------------------------------------------------------
%% Section
\section{Conclusão }
A utilização de recursos da programação nos permite, criar  de programas simples a jogos complexos,além de nos ajudar a realizar nossos trabalhos de forma mais rápida e eficiente. A partir desse projeto foi possivel aprender na prática toda a teoria passada durante as aulas de Programação I, e a buscar conhecimentos além do que nos foi imposto.







%%------------------------------------------------------
%% Section (no numbering, use section* for acknowledgment)
%% UFSC: no necessary
% \section*{Acknowledgment}
% The authors would like to thank...



%%------------------------------------------------------
%% References (Option 1): extern file
%% Edit with JabRef, for instance
\bibliographystyle{IEEEtran} %% Estilo da referência
\bibliography{references}    %% Caminho do arquivo (sem extensão)



% %%------------------------------------------------------
% %% References (Option 2): incorporeted
% 
% \begin{thebibliography}{10} 
% 
%   \bibitem{Kopka:1999}
%     H.~Kopka and P.~W. Daly, \emph{A Guide to
%     \LaTeX}, 3rd~ed. Harlow, 
%     England: Addison-Wesley, 1999.
% 
%   \bibitem{Huynen:1998} 
%     Huynen, M.~A. and Bork, P. 1998. Measuring 
%     genome evolution. {\em Proceedings of the 
%     National Academy of Sciences USA} 
%     95:5849--5856.
% 
%   \bibitem{Caprara:1997} 
%     Caprara, A. 1997. Sorting by reversals is 
%     difficult. In: {\em Proceedings of the 
%     First Annual International Conference on 
%     Computational Molecular Biology 
%     (RECOMB 97),} New York: ACM.  pp. 75-83.
% 
%   \bibitem{Reinelt:1991}
%     Reinelt, G. 1991. {\em The Traveling 
%     Salesman -- Computational Solutions for TSP 
%     Applications.} Berlin: Springer Verlag.
% 
% \end{thebibliography}



\end{document}